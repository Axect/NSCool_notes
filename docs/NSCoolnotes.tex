\documentclass[]{book}
\usepackage{lmodern}
\usepackage{amssymb,amsmath}
\usepackage{ifxetex,ifluatex}
\usepackage{fixltx2e} % provides \textsubscript
\ifnum 0\ifxetex 1\fi\ifluatex 1\fi=0 % if pdftex
  \usepackage[T1]{fontenc}
  \usepackage[utf8]{inputenc}
\else % if luatex or xelatex
  \ifxetex
    \usepackage{mathspec}
  \else
    \usepackage{fontspec}
  \fi
  \defaultfontfeatures{Ligatures=TeX,Scale=MatchLowercase}
\fi
% use upquote if available, for straight quotes in verbatim environments
\IfFileExists{upquote.sty}{\usepackage{upquote}}{}
% use microtype if available
\IfFileExists{microtype.sty}{%
\usepackage{microtype}
\UseMicrotypeSet[protrusion]{basicmath} % disable protrusion for tt fonts
}{}
\usepackage[margin=1in]{geometry}
\usepackage{hyperref}
\hypersetup{unicode=true,
            pdftitle={NSCool-notes},
            pdfauthor={Tae Geun Kim},
            pdfborder={0 0 0},
            breaklinks=true}
\urlstyle{same}  % don't use monospace font for urls
\usepackage{natbib}
\bibliographystyle{apalike}
\usepackage{longtable,booktabs}
\usepackage{graphicx,grffile}
\makeatletter
\def\maxwidth{\ifdim\Gin@nat@width>\linewidth\linewidth\else\Gin@nat@width\fi}
\def\maxheight{\ifdim\Gin@nat@height>\textheight\textheight\else\Gin@nat@height\fi}
\makeatother
% Scale images if necessary, so that they will not overflow the page
% margins by default, and it is still possible to overwrite the defaults
% using explicit options in \includegraphics[width, height, ...]{}
\setkeys{Gin}{width=\maxwidth,height=\maxheight,keepaspectratio}
\IfFileExists{parskip.sty}{%
\usepackage{parskip}
}{% else
\setlength{\parindent}{0pt}
\setlength{\parskip}{6pt plus 2pt minus 1pt}
}
\setlength{\emergencystretch}{3em}  % prevent overfull lines
\providecommand{\tightlist}{%
  \setlength{\itemsep}{0pt}\setlength{\parskip}{0pt}}
\setcounter{secnumdepth}{5}
% Redefines (sub)paragraphs to behave more like sections
\ifx\paragraph\undefined\else
\let\oldparagraph\paragraph
\renewcommand{\paragraph}[1]{\oldparagraph{#1}\mbox{}}
\fi
\ifx\subparagraph\undefined\else
\let\oldsubparagraph\subparagraph
\renewcommand{\subparagraph}[1]{\oldsubparagraph{#1}\mbox{}}
\fi

%%% Use protect on footnotes to avoid problems with footnotes in titles
\let\rmarkdownfootnote\footnote%
\def\footnote{\protect\rmarkdownfootnote}

%%% Change title format to be more compact
\usepackage{titling}

% Create subtitle command for use in maketitle
\newcommand{\subtitle}[1]{
  \posttitle{
    \begin{center}\large#1\end{center}
    }
}

\setlength{\droptitle}{-2em}

  \title{NSCool-notes}
    \pretitle{\vspace{\droptitle}\centering\huge}
  \posttitle{\par}
    \author{Tae Geun Kim}
    \preauthor{\centering\large\emph}
  \postauthor{\par}
      \predate{\centering\large\emph}
  \postdate{\par}
    \date{2019-01-31}

\usepackage{booktabs}
\usepackage{amsthm}
\makeatletter
\def\thm@space@setup{%
  \thm@preskip=8pt plus 2pt minus 4pt
  \thm@postskip=\thm@preskip
}
\makeatother

\begin{document}
\maketitle

{
\setcounter{tocdepth}{1}
\tableofcontents
}
\hypertarget{preface}{%
\chapter{Preface}\label{preface}}

This note is summary to implement neutron star cooling.

\hypertarget{TOV}{%
\chapter{TOV Equation}\label{TOV}}

\begin{itemize}
\tightlist
\item
  TOV equation means \textbf{Tolman-Oppenheimer-Volkoff equation}.
\end{itemize}

\hypertarget{build-equation}{%
\section{Build Equation}\label{build-equation}}

For static, non-rotating and spherical symmetric star, metric is given as
\[ds^2 = -e^{2\Phi(r)}dt^2 + e^{2\Lambda(r)} dr^2 + r^2 d\Omega^2\]

For isolated star, the metric must reduce to the Schwarzschild metric at outside of the star
\[ds^2 = -\left(1 - \frac{2M}{r}\right) dt^2 + \frac{1}{1 - \frac{2M}{r}} dr^2 + r^2 d\Omega^2\]

At interior of the star, let define new metric function for convenience
\[e^{-2\Lambda(r)} = 1 - \frac{2m(r)}{r}\]

The quantuty \(m(r)\) can be interpreted as \textbf{the mass interior to radius \(r\)}.

Now, consider perfect fluid matter:
\[T_{\mu\nu} = (\rho + P)u_\mu u_\nu + Pg_{\mu\nu}, ~u = e^{-\Phi}\frac{\partial}{\partial t}\]

Rewrite this to index-free notation:
\[T = \rho^{2\Phi} dt^2 + \frac{P}{1 - \frac{2m(r)}{r}}dr^2 + Pr^2 d\Omega^2\]

Solve Einstein equation with this metric \& EM tensor then,
\begin{align}
    \frac{dm(r)}{dr} &= 4\pi r^2 \rho \\
    \frac{d\Phi(r)}{dr} &= \frac{m(r) + 4\pi r^3 P}{r^2 \left(1 - \frac{2m(r)}{r}\right)}
\end{align}

From energy conservation, we can get one more equation:
\[\nabla_\mu {T^\mu}_\nu = 0 \Rightarrow (\rho + P) \frac{d\Phi}{dr} + \frac{dP}{dr} = 0\]

These three fundamental equations are called \textbf{Tolman-Oppenheimer-Volkoff Equations}

\begin{longtable}[]{@{}l@{}}
\toprule
\endhead
\begin{minipage}[t]{0.97\columnwidth}\raggedright
\textbf{Tolman-Oppenheimer-Volkoff equation}\strut
\end{minipage}\tabularnewline
\begin{minipage}[t]{0.97\columnwidth}\raggedright
\[\begin{aligned}
\frac{dm}{dr} &= 4\pi r^2 \rho \\
\frac{d\Phi}{dr} &= \frac{ m + 4\pi r^3 P}{r^2 (1 - \frac{2m}{r})} \\
\frac{dP}{dr} &= - \frac{(\rho + P)(m + 4\pi r^3 P)}{r^2 (1 - \frac{2m}{r})}
\end{aligned}\]\strut
\end{minipage}\tabularnewline
\bottomrule
\end{longtable}

~

Now, use polytropic equation of state.

\begin{longtable}[]{@{}l@{}}
\toprule
\endhead
\begin{minipage}[t]{0.97\columnwidth}\raggedright
\textbf{Polytropic Equation of State}\strut
\end{minipage}\tabularnewline
\begin{minipage}[t]{0.97\columnwidth}\raggedright
\[P = K \rho_0^\Gamma\]
\(P\): pressure
\(K\): polytropic gas constant
\(\rho_0\): rest mass density. for neutron star, \(\rho_0 = m_u n_B\)
\(m_u\): nucleon mass with atomic mass unit
\(n_B\): baryonic number density
\(\rho = \rho_0(1 + \epsilon)\): total mass energy density
\(\epsilon\): internal energy density per unit mass
\(\Gamma \equiv 1 + \frac{1}{n}\)
\(n\): polytropic index\strut
\end{minipage}\tabularnewline
\bottomrule
\end{longtable}

Thermodynamics process of neutron star is isentropic process.
It means \(S, N\) are conserved and only \(V\) changes.
First define densities as belows.
\[s = \frac{S}{V},~ n = \frac{N}{V},~ \rho = \frac{U}{V}\]
Since \(dS = dN = 0\), we can get next relations.
\[\frac{ds}{s} = \frac{dn}{n} = - \frac{dV}{V}\]
Now, denote 1st law of thermodynamics.
\[dU = -PdV + TdS + \mu dN = -PdV\]
Rewrite this in terms of \(\rho\), then we can get next relation.
\[\frac{d\rho}{\rho + P} = - \frac{dV}{V}\]
By above two relations, next equation is naturally given.
\[\left(\rho + P\right) = n \frac{\partial \rho}{\partial n}\]
Since, \(\rho = \rho_0 (1+\epsilon),~ \rho_0 = m_u n\),
\[\frac{\partial \rho}{\partial n} = \frac{\rho}{n} + \frac{\partial}{\partial n}\left(\frac{\rho}{n}\right) \cdot n\]
Then finally we can obtain next equation.
\[\therefore \, P = n^2 \frac{\partial}{\partial n} \left(\frac{\rho}{n}\right)\]

Let substitute \(P\) from polytropic EoS, then
\[\rho = \rho_0 + \frac{P}{\Gamma-1}\]

\begin{longtable}[]{@{}l@{}}
\toprule
\endhead
\begin{minipage}[t]{0.97\columnwidth}\raggedright
\textbf{Total Running Terms }\strut
\end{minipage}\tabularnewline
\begin{minipage}[t]{0.97\columnwidth}\raggedright
\[\begin{aligned}
\frac{dm}{dr} &= 4\pi r^2 \rho \\
\frac{d\Phi}{dr} &= \frac{ m + 4\pi r^3 P}{r^2 (1 - \frac{2m}{r})} \\
\frac{dP}{dr} &= - \frac{(\rho + P)(m + 4\pi r^3 P)}{r^2 (1 - \frac{2m}{r})} \\
\frac{dN_B}{dr} &= \frac{4\pi r^2 n_B}{\sqrt{1 - \frac{2m}{r}}}
\end{aligned}\]
where
\[\begin{gathered}
\rho_0 = \left(\frac{P}{K}\right)^{\frac{1}{\Gamma}} \\
\rho = \rho_0 + \frac{P}{\Gamma - 1} \\
n_B = \frac{1}{m_u} \left(\frac{P}{K}\right)^{\frac{1}{\Gamma}}
\end{gathered}\]\strut
\end{minipage}\tabularnewline
\bottomrule
\end{longtable}

\begin{longtable}[]{@{}l@{}}
\toprule
\endhead
\begin{minipage}[t]{0.97\columnwidth}\raggedright
\textbf{Boundary Conditions}\strut
\end{minipage}\tabularnewline
\begin{minipage}[t]{0.97\columnwidth}\raggedright
\[\begin{aligned}
m(0) &= 0\\
\Phi(0) &= 0\\
\rho_0(0) &= \rho_c \\
P(0) &= P_c = K\rho_c^{\Gamma}\\
n_B(0) &= n_c = \frac{1}{m_u} \rho_c \\
P(R) &= 0 \\
\Phi(R) &= \frac{1}{2}\ln \left(1 - \frac{2GM}{R}\right)
\end{aligned}\]\strut
\end{minipage}\tabularnewline
\bottomrule
\end{longtable}

\bibliography{book.bib}


\end{document}
